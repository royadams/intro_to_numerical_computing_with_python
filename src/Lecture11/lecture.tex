\documentclass[serif,xcolor=pdftex,dvipsnames,table,hyperref={bookmarks=false,breaklinks}]{beamer}

%%%%%%%%%%%%%%%%
% Change the macros below to configure the title slides
% for your course.
\newcommand{\coursename}{COMPSCI 590N}
\newcommand{\instructor}{Roy J. Adams}
\newcommand{\university}{University of Massachusetts Amherst}
\newcommand{\department}{College of Information and Computer Sciences}
%%%%%%%%%%%%%%%%

\newcommand\HUGE{\@setfontsize\Huge{50}{60}}

\newcommand{\settitlecard}[2]{
  \title[\coursename  Lecture #1] 
    {\coursename \\ Lecture #1: #2}
     \author[\instructor]{\instructor}
     \institute[\university]{
     \department\\
     \university
   }
\date{}
}

\newcommand{\maketitlepage}{
  \begin{frame}
  \titlepage
  %\center{
    %If you use the slides unmodified, retain the attribution below
  %  \tiny{Slides by Roy J. Adams (rjadams@cs.umass.edu). 
    %If you modify the slides, please retain the alternate attribution below
    %\tiny{Based on slides by Roy J. Adams (rjadams@cs.umass.edu). \\    
  %  }                                              
  %}  
  \end{frame}
}

\AtBeginSection[]
{
  \begin{frame}<beamer>{Outline}
    \tableofcontents[currentsection,subsectionstyle=hide]
  \end{frame}
}


\newcommand{\cut}[1]{}

\newcommand{\iconbox}[4]{
  \only<#1-#2>{
    \begin{columns}[T]
      \column{0.5in}
           \includegraphics[width=0.5in]{#3}
       \column{3.7in}
            #4
    \end{columns}
    \medskip
    \medskip
    \medskip
  }
}

\mode<presentation>{
  \usepackage{../beamertheme589theme}
  \setbeamercovered{invisible}
}

\mode<handout>{
  \usepackage{../beamertheme589theme}
  \setbeamercovered{transparent}
}


\usepackage[english]{babel}
\usepackage[latin1]{inputenc}
\usepackage{times}
\usepackage[T1]{fontenc}
\usepackage{amsmath}
\usepackage{amssymb}
\usepackage[noend]{algorithmic}
\usepackage{algorithm}
\usepackage{listings}
\usepackage{tcolorbox}
\usepackage{xmpmulti}

\renewcommand\mathfamilydefault{\rmdefault}

\newcommand{\setA}{\mathcal{A}}
\newcommand{\setB}{\mathcal{B}}
\newcommand{\setS}{\mathcal{S}}
\newcommand{\setV}{\mathcal{V}}
\DeclareMathOperator*{\union}{\bigcup}
\DeclareMathOperator*{\intersection}{\bigcap}
\DeclareMathOperator*{\Val}{Val}
\newcommand{\mbf}[1]{{\mathbf{#1}}}
\DeclareMathOperator*{\argmax}{arg\,max}
\DeclareMathOperator*{\argmin}{arg\,min}
\DeclareMathOperator*{\sign}{sign}
\newcommand{\deriv}[2]{\frac{\partial{#1}}{\partial{#2}}}

\lstdefinestyle{custompython}{
  belowcaptionskip=1\baselineskip,
  breaklines=true,
  frame=single,
  xleftmargin=\parindent,
  language=Python,
  showstringspaces=false,
  basicstyle=\footnotesize\ttfamily,
  keywordstyle=\bfseries\color{green!40!black},
  commentstyle=\itshape\color{purple!40!black},
  identifierstyle=\color{blue},
  stringstyle=\color{orange},
}
\lstset{style=custompython}

\makeatletter
\renewcommand*\env@matrix[1][*\c@MaxMatrixCols c]{%
  \hskip -\arraycolsep
  \let\@ifnextchar\new@ifnextchar
  \array{#1}}
\makeatother

\newcommand\norm[1]{\left\lVert#1\right\rVert}


\settitlecard{11}{Debugging and Exceptions}

\begin{document}

\maketitlepage

% \section{Announcements}
% \subsection{Foo}
%
% \begin{frame}[t]{Announcements}
% \begin{itemize}
% 	\item Quiz 5 due tonight.
% 	\item Assignment 5 due Tuesday.
% \end{itemize}
% \end{frame}
%
% \section{Quiz 3}
% \subsection{Foo}
%
% \begin{frame}[t,fragile]{Quiz 3}
% 	\begin{lstlisting}
% 		# O(n^2) algorithm
% 		for i in range(n):
% 		    for j in range(n):
% 		        simple_operation()
%
% 		# O(n) algorithm
% 		for i in range(n):
% 		    expensive_operation()
% 	\end{lstlisting}
% \end{frame}

\section{Debugging}
\subsection{Foo}

\begin{frame}[t]{Debugging}
	\begin{itemize}[<+->]
		\item \textbf{Debugging} is the process of finding the errors in your code.
		\item This process can take as much or more time as writing the code itself.
		\item Having skill at debugging can save you \textbf{lots} of time.
		\item Most programming languages have tools built for debugging code, aptly called \textbf{debuggers}.
		\item Python has a number of these:
		\begin{itemize}[<+->]
			\item \textbf{pdb} - the built-in debugger.
			\item \textbf{DDD} - graphical front-end for many debuggers including pdb.
			\item \textbf{Spyder} - Spyder has a built in debugger.
			\item \textbf{PuDB} - another graphical debugger.
		\end{itemize}
	\end{itemize}
\end{frame}

\begin{frame}[t]{Debugging}
	The methods you use for debugging will vary depending on what kind of code you are writing, but there are some guiding principles.
	\pause
	\begin{block}{The Principle of Confirmation}
		``Fixing a buggy program is a process of confirming, one by one, that the many things you believe to be true about the code actually \emph{are} true. When you find that one of your assumptions is not true, you have found a clue to the location (if not the exact nature) of a bug.''
		
		- \emph{Fast Lane to Python}, Norm Matloff
	\end{block}
	
	\pause
	In other words we are looking for things we don't expect.
\end{frame}

\begin{frame}[t]{Debugging Strategy}
	There are two main steps to debugging:
	\pause
	\begin{enumerate}[<+->]
		\item \textbf{Locate} the bug.
		\item \textbf{Understand} the bug.
	\end{enumerate}
	
	\pause 
	\begin{itemize}[<+->]
		\item A very common mistake made by new programmers is to try and figure out what went wrong before figuring out where it went wrong. 
		\item Localizing a bug makes understanding the bug \textbf{much} easier.
	\end{itemize}
\end{frame}

% \begin{frame}[t]{Debugging by Print Statements}
% 	% Confirmation Example
% 	For example:
% 	\begin{itemize}
% \end{frame}

\begin{frame}[t,fragile]{Types of Bugs: Syntax vs Semantics}
	There are many types of bugs. One common distinction is a syntax vs a semantic error.
	\begin{itemize}[<+->]
		\item A \textbf{syntax error} is caused by typing code that the computer cannot understand.
		\item For example:
	\end{itemize}
	\pause
	\begin{lstlisting}
		>>> 1 !== 2
		  File "<stdin>", line 1
		    1 !== 2
		        ^
		SyntaxError: invalid syntax
	\end{lstlisting}
	\pause
	\begin{itemize}
		\item These will almost always cause your program to crash and are usually easy to find.
	\end{itemize}
\end{frame}

\begin{frame}[t]{Types of Bugs: Syntax vs Semantics}
	\begin{itemize}
		\item These will almost always cause your program to crash and are usually easy to find.
		\item A \textbf{semantic error} occurs when the code is valid, but does not do what you expect it to.
		\item These can be very difficult to find.
	\end{itemize}
\end{frame}

\begin{frame}[t]{Types of Bugs: Deterministic vs Non-deterministic}
	\begin{itemize}[<+->]
		\item A \textbf{deterministic} bug is one that will appear every time the same input is used.
		\item This is the most common.
		\item Still hard to find because they may not appear for every input.
		\item A \textbf{non-deterministic} bug is one that appears irregularly.
		\item Common causes include:
		\begin{itemize}[<+->]
			\item Code that relies on randomness (e.g. samplers).
			\item Parallel code with race conditions.
			\item Code that relies of exterior state.
		\end{itemize}
		\item The most nefarious of these is the so-called \textbf{heisenbug} which disappears when you try to view it.
	\end{itemize}
\end{frame}

\begin{frame}[t]{Debugging by Print Statements}
	\begin{itemize}[<+->]
		\item One very common method for accessing the intermediate state of a program is to print the values of variables or add print statements inside conditional blocks.
		\item We all do it, but we really shouldn't...
		\item As a general rule, you should not modify your code in order to debug it.
		\item In some cases modifying the code can make it so you can't reproduce the bug.
		\item Sometimes unavoidable.
	\end{itemize}
\end{frame}

% \begin{frame}[t]{Debugging Tools}
% 	\begin{itemize}[<+->]
% 		\item
% 	\end{itemize}
% \end{frame}

\begin{frame}[t,fragile]{pdb}
	\begin{itemize}[<+->]
		\item Instead you should use one of the many debugging tools.
		\item The built-in Python debugger is called \verb|pdb|.
	\end{itemize}
	% Calling pdb
	\pause
	\begin{lstlisting}
		$> python -m pdb script.py
		> /Users/adams/Dropbox/590n/demos/script.py(1)<module>()
		-> import numpy as np
		(Pdb) 
	\end{lstlisting}
\end{frame}

\begin{frame}[t,fragile]{Breakpoints}
	\begin{itemize}[<+->]
		\item \textbf{Breakpoints} allow you to stop the code at a specified line and interactively examine the state.
		\item In pdb, breakpoints are set with the command \verb|b| or \verb|break| followed by a line number.
	\end{itemize}
	\pause
	\begin{lstlisting}
		(Pdb) b 8
		Breakpoint 1 at /Users/adams/Dropbox/590n/demos/script.py:8
		(Pdb)
	\end{lstlisting}
	\pause
	\begin{itemize}[<+->]
		\item To run the script, use the \verb|c| or \verb|continue| command.
	\end{itemize}
	\pause
	\begin{lstlisting}
		(Pdb) c
		> /Users/adams/Dropbox/590n/demos/script.py(8)<module>()
		-> b = 3
		(Pdb)
	\end{lstlisting}

\end{frame}

\begin{frame}[t,fragile]{Viewing Variables}
	\begin{itemize}[<+->]
		\item Once you are inside the code, you can explore the state (that is, the current values of the variables).
		\item Print variables using the \verb|p| or \verb|print| command.
	\end{itemize}
	\pause
	\begin{lstlisting}
		(Pdb) c
		> /Users/adams/Dropbox/590n/demos/script.py(8)<module>()
		-> b = 3
		(Pdb) p a
		2
	\end{lstlisting}
	
\end{frame}

\begin{frame}[t,fragile]{Conditional Breakpoints}
	
	\begin{itemize}[<+->]
		\item It is often convenient to view the state only when a certain condition is met.
		\item For example, if we want to stop inside a loop but only on the last iteration.
		\item A \textbf{conditional breakpoint} will halt the code only when a logical condition returns True.
		\item This condition may involve variables defined in the code.
	\end{itemize}
	\pause
	\begin{lstlisting}
		# Break at line <line_number> when <condition> is True
		(Pdb) b <line_number>, <condition>
	\end{lstlisting}
	
\end{frame}

\begin{frame}[t,fragile]{Stepping Through Code}
	% step next until
	\begin{itemize}[<+->]
		\item Once you have reached a breakpoint, it is often useful to step through the code one line at a time.
		\item The \verb|n| or \verb|next| command will advance the code to the next line \textbf{in the current scope}. It will not enter functions calls.
		\item The \verb|s| or \verb|step| command will advance the code to the next line and will enter function calls.
		\item The \verb|until| command will run the code until the specified line.
	\end{itemize}
	\pause
	\begin{lstlisting}
		(Pdb) s
		(Pdb) n
		(Pdb) until <line_number>
	\end{lstlisting}
\end{frame}

\section{Exception Handling}
\subsection{Foo}

\begin{frame}[t,fragile]{Exceptions}
	\begin{itemize}[<+->]
		\item When Python encounters a run time error, it raises an \textbf{exception}.
		\item By default this will crash your code.
		\item There are many cases where we might want to keep running:
		\begin{itemize}[<+->]
			\item Continuous code (e.g. a server application).
			\item Restart algorithm with a new initialization.
			\item Solicit the user for input.
		\end{itemize}
	\end{itemize}
\end{frame}

\begin{frame}[t,fragile]{Exceptions}
	\begin{itemize}[<+->]
		\item The Python \textbf{try-except} block allows you to gracefully recover from an error.
	\end{itemize}
	\pause
	\begin{lstlisting}
		try:
		    <code_that_might_have_errors>
		except:
		    <error_handling_code>
	\end{lstlisting}
	\pause
	\begin{itemize}[<+->]
		\item Python will try to run the code in the try block and if it fails, it will run the code in the except block.
	\end{itemize}
\end{frame}

\begin{frame}[t,fragile]{Exceptions}
	\begin{itemize}[<+->]
		\item Sometimes errors contain information and you may want to handle different exceptions in different ways.
	\end{itemize}
	\pause
	\begin{lstlisting}
		try:
		    <code_that_might_have_errors>
		except <exception_type1>:
		    <type1_handler>
		except <exception_type2>:
		    <type2_handler>
		except:
		    <handle_everything_else>
	\end{lstlisting}
\end{frame}

\begin{frame}[t,fragile]{Exceptions}
	\begin{itemize}[<+->]
		\item Common built-in exception types include:
		\begin{itemize}[<+->]
			\item \verb|ValueError|
			\item \verb|ArithmeticError|
			\item \verb|IOError|
			\item \verb|KeyError|
			\item etc.
		\end{itemize}
	\end{itemize}
\end{frame}

\begin{frame}[t,fragile]{Exceptions}
	% Example
	You can handle different errors in different ways:
	\begin{lstlisting}
		try:
		    <code_that_might_have_errors>
		except <exception_type1>:
		    <type1_handler>
		except <exception_type2>:
		    <type2_handler>
		except:
		    <handle_everything_else>
	\end{lstlisting}
\end{frame}

\begin{frame}[t,fragile]{Raising Exceptions}
	\begin{itemize}[<+->]
		\item Python also allows you to \textbf{raise} your own exceptions.
		\item Reasons you may want to do this include:
		\begin{itemize}[<+->]
			\item It provides information about the error (as opposed to failing silently).
			\item Allows the person using your code to handle the exception gracefully.
			\item You will probably forget all of the ways your code can fail...
		\end{itemize}
		\item Raise exceptions using the \verb|raise| command.
	\end{itemize}
	\pause
	\begin{lstlisting}
		# Raise a value error
		raise ValueError("Cannot invert a singular matrix.")
	\end{lstlisting}
\end{frame}

\begin{frame}[t,fragile]{Cleaning Up}
	\begin{itemize}[<+->]
		\item Sometimes there is code you want to run regardless of whether there was an error or not.
		\item This code can be written in a \textbf{finally} clause.
		\item Code in a finally block with run whether or not the code in the try block succeeds.
	\end{itemize}
	\pause
	\begin{lstlisting}
		try:
		    <code_that_might_have_errors>
		except:
		    <error_handling_code>
		finally:
		    <always_run_this>
	\end{lstlisting}
\end{frame}

\begin{frame}[t,fragile]{Exception Handling is Not Just For Exceptions}
	% Example
	Try-except blocks can be used for many things besides handling errors. For example:
	\pause
	\begin{lstlisting}
		a = {}
		try:
		    a["key"].append(5)
		except:
		    a["key"] = [5]
	\end{lstlisting}
	\pause
	\begin{itemize}
		\item Here we are try to create a dictionary of lists.
		\item We can use try-except to check if a list is in the dictionary and add it if it is not.
	\end{itemize}
\end{frame}

% \section{Unit Testing}
% \subsection{Foo}
%
% \begin{frame}[t,fragile]{Testing Code}
%
% \end{frame}

\end{document}
